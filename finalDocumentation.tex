\documentclass[11pt]{article}
\usepackage{amsmath}
\usepackage{graphicx}


\begin{document}
\tableofcontents
	
\title{Cheers}

\newpage
\section{Introduction}
	Let’s consider two identical circular coasters, with negligible heights. We place them partially overlapping one another. We are interested in finding, how much the upper coaster has to be moved over the lower coaster, so that the area of the overlapping region is half of any one coaster.
		
\section{Problem Statement}
	Coaster is an item on which, one can place the glass of a beverage or other drink. Coasters protect the surface of a table or any other surface where the one might place a beverage. They are used not just to protect the surface of the table, but, as they are usually made of paper, they can also be used to absorb condensation dripping along the glass or serve as an ad-hoc notepad. [1] The tegestologists are also fascinated in collecting the coasters. 
	
\section{Soution}
	Let’s consider two identical circular coasters, with negligible heights. We place them partially overlapping one another. We are interested in finding, how much the upper coaster has to be moved over the lower coaster, so that the area of the overlapping region is half of any one coaster as shown in the figure 1.
	Then the length $l$ of the segment $X_1X_2$ is given by the equation,\\
	
	 $$l = 2R(1 – \cos(\alpha/2))$$,\\
	
	and $\alpha$ is given by the equation\\
	
	 $$\alpha – \sin(\alpha) = \pi/2$$.
	
\end{document}
